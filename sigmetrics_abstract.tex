\documentclass[sigconf]{acmart}
%\documentclass{sig-alternate-05-2015}

\usepackage{amsmath}
\usepackage{amsfonts}
\usepackage[english]{babel}
%\usepackage[backend=bibtex,firstinits=true,maxbibnames=99]{biblatex}
%\bibliography{biblio}
\usepackage{tikz}
\usepackage[utf8]{inputenc}
\usepackage{calc}

\usepackage[subject={Todo},author={Nicolas}]{pdfcomment}

\graphicspath{{}{simu/}{jsq2_simulate/}}

\usepackage{hyperref}
\definecolor{darkblue}{rgb}{0 0 .6}
\hypersetup{colorlinks=true,linkcolor=darkblue,citecolor=darkblue,urlcolor=darkblue}
\hypersetup{pageanchor=false}
\newcommand\SN{S^{(N)}}
\newcommand\XN{X^{(N)}}
\newcommand\XtN{\tilde{X}^{(N)}}
\newcommand\YN{Y^{(N)}}
\newcommand\ZN{Z^{(N)}}
\newcommand\MN{M^{(N)}}
\newcommand\LN{L^{(N)}}
\newcommand\fN{f^{(N)}}
\newcommand\betaN{\beta^{(N)}}
\newcommand\PsiN{\Psi^{(N)}}
\newcommand\E{\mathcal{E}}
\newcommand\N{\mathbb{N}}
\newcommand\R{\mathbb{R}}
\newcommand\Z{\mathbb{Z}}
\newcommand\bm{\mathbf{m}}
\newcommand\bM{\mathbf{M}}
\newcommand\bx{\mathbf{x}}
\newcommand\bc{\mathbf{c}}
\newcommand\calT{\mathcal{T}}
\newcommand\calV{\mathcal{V}}
\newcommand\calL{\mathcal{L}}
\newcommand\calC{\mathcal{C}}
\newcommand\calF{\mathcal{F}}
\newcommand\calP{\mathcal{P}}
\newcommand\calS{\mathcal{S}}
\newcommand\calB{\mathcal{B}}
\newcommand\calA{\mathcal{A}}
\newcommand\calN{\mathcal{N}}
\newcommand\floor[1]{\left\lfloor#1\right\rfloor}
\newcommand\var[1]{\mathrm{var}\left[#1\right]}
\newcommand\svar[1]{\mathrm{var}[#1]}
\newcommand\esp[1]{{\mathchoice{\besp{#1}}{\sesp{#1}}{\sesp{#1}}{\sesp{#1}}}}
\newcommand\besp[1]{\mathbb{E}\left[#1\right]}
\newcommand\sesp[1]{\mathbb{E}[#1]}
\newcommand\espN[1]{{\mathchoice{\bespN{#1}}{\sespN{#1}}{\sespN{#1}}{\sespN{#1}}}}
\newcommand\bespN[1]{\mathbf{E}^{(N)}\left[#1\right]}
\newcommand\sespN[1]{\mathbf{E}^{(N)}[#1]}
\newcommand\Proba[1]{{\mathchoice{\bProba{#1}}{\sProba{#1}}{\sProba{#1}}{\sProba{#1}}}} 
\newcommand\bProba[1]{\mathbf{P}\left[#1\right]}
\newcommand\sProba[1]{\mathbf{P}[#1]}
\newcommand\norm[1]{{\mathchoice{\bnorm{#1}}{\snorm{#1}}{\snorm{#1}}{\snorm{#1}}}}
\newcommand\bnorm[1]{\left\|#1\right\|}
\newcommand\snorm[1]{\|#1\|}
\newcommand\norminf[1]{\left\|#1\right\|_{\infty}}
\newcommand\abs[1]{\left|#1\right|}
\newcommand\Ind[1]{\mathbf{1}_{\{#1\}}}
\newcommand\F{\mathcal{F}}
\newcommand\dt{\frac{d}{dt}}
\newcommand\pder[1]{\frac{\partial }{\partial#1}}
\newcommand\red[1]{{\color{red}#1}}
\newcommand\p[1]{\left(#1\right)}
\newcommand\bbm{\mathbf{m}}
\DeclareMathOperator*{\argmin}{arg\,min}
\newcommand\proba[1]{\mathbb{P}\left(#1\right)}

\newcommand\myComment[1]{ \pdfmargincomment[color=red]{#1}}
\newcommand\TBC{\myComment{to be checked}}

%\newtheorem{definition}{Definition}
%\newtheorem{theorem}{Theorem}
%\newtheorem{lemma}{Lemma}

\newtheorem{coro}{Corollary}

% Copyright
\copyrightyear{2017} 
\acmYear{2017} 
\setcopyright{rightsretained} 
\acmConference{SIGMETRICS '17}{June 05-09, 2017}{Urbana-Champaign, IL,
  USA}
\acmDOI{10.1145/3078505.3078523}
\acmISBN{978-1-4503-5032-7/17/06}

\newcommand\githublink{\url{https://github.com/ngast/meanFieldAccuracy}}

\title{Expected Values Estimated via Mean-Field Approximation are
  1/N-Accurate}%
\subtitle{Extended Abstract}%
\subtitlenote{Full version of the paper : \githublink{}
  \cite{gast2016_1/N}.}

\author{Nicolas Gast}
\orcid{0000-0001-6884-8698}%
\affiliation{%
  \institution{Inria}%
  \streetaddress{Univ. Grenoble Alpes, CNRS, LIG}%
  \city{ Grenoble} %
  \state{France} %
  \postcode{F-38000}%
}%
\email{nicolas.gast@inria.fr}

\date{\today}

\begin{document}

\begin{abstract}
  In this paper, we study the accuracy of mean-field approximation.
  We show that, under general conditions, the expectation of any
  performance functional converges at rate $O(1/N)$ to its mean-field
  approximation.  Our result applies for finite and
  infinite-dimensional mean-field models.  We provide numerical
  experiments that demonstrate that this rate of convergence is tight.
\end{abstract}

\begin{CCSXML}
<ccs2012>
<concept>
<concept_id>10002944.10011123.10011674</concept_id>
<concept_desc>General and reference~Performance</concept_desc>
<concept_significance>500</concept_significance>
</concept>
<concept>
<concept_id>10002950.10003648.10003700</concept_id>
<concept_desc>Mathematics of computing~Stochastic processes</concept_desc>
<concept_significance>500</concept_significance>
</concept>
<concept>
<concept_id>10002950.10003648.10003688.10003689</concept_id>
<concept_desc>Mathematics of computing~Queueing theory</concept_desc>
<concept_significance>300</concept_significance>
</concept>
<concept>
<concept_id>10002950.10003714.10003727.10003728</concept_id>
<concept_desc>Mathematics of computing~Ordinary differential equations</concept_desc>
<concept_significance>100</concept_significance>
</concept>
</ccs2012>
\end{CCSXML}

\ccsdesc[500]{General and reference~Performance}
\ccsdesc[500]{Mathematics of computing~Stochastic processes}
\ccsdesc[300]{Mathematics of computing~Queueing theory}
\ccsdesc[100]{Mathematics of computing~Ordinary differential equations}
% We no longer use \terms command
%\terms{Theory}

\keywords{Mean-field approximation; queueing theory; accuracy of
  approximation; supermarket model; power-of-two-choice}

\maketitle

\section{Introduction}

Mean-field approximation is a powerful tool for studying systems
composed of a large number of interacting objects. The idea of
mean-field approximation is to replace a complex stochastic system by
a simpler deterministic dynamical system.  This approximation is
widely used to study the performance of computer-based systems.
This approximation is known to be asymptotically exact for many
systems, in which the fraction of objects in a given state $i$,
$\XN_i$, converges at rate $O(1/\sqrt{N})$ to a deterministic quantity
$x_i$, as the number of objects $N$ goes to infinity
\cite{kurtz70,ying2016rate}.


In this paper, we show that, when one wants to estimate expected
values, the rate of convergence is much faster. More precisely, we
show that for a smooth function $h$, the expectation $\esp{h(\XN)}$
converges at rate $1/N$ to its mean-field approximation $h(x)$:
\begin{align}
  \label{eq:1/N}
  \abs{\esp{h(\XN)} - h(x)} = O\p{\frac1N}.
\end{align}
We show that this essentially holds for the transient regime as soon
as the drift of the system is twice-differentiable. It holds for the
stationary regime if in addition the differential equation has a
unique stable point that is exponentially stable. We also exhibit an
example that shows that when the drift of the system is only
Lipschitz-continuous, the convergence can be slower.

As an example, we study in detail the convergence rate of the
classical power-of-two-choice model (\emph{a.k.a.} supermarket model)
of \cite{mitzenmacher1996power,vvedenskaya1996queueing}. These papers
show that, as the number of servers $N$ go to infinity, the average
queue length goes to $m^{\infty}(\rho)=\Theta(\log 1/(1-\rho))$.

Our results show that an average value estimated via mean-field
approximation is $1/N$-accurate.  In a queuing network such as the
two-choice model, the average queue length can be expressed as
$\sesp{h(\XN)}$.  Equation~\eqref{eq:1/N} shows that the average queue
length converges at rate $O(1/N)$ to its mean-field approximation.  We
provide numerical evidence that for this model, for any finite value
of $N$, the average queue length is
$ m^N(\rho) \approx m^{\infty}(\rho)+\frac{\rho^2}{2N(1-\rho)}$. We
illustrate this in Table~\ref{tab:power2} where we report the average
queue length for $\rho=0.9$. We observe that
$m^{N}(0.9)\approx m^{\infty}(0.9)+4/N$.

\begin{table}[ht]
  \centering
  \begin{tabular}{|c|c|c|c|c|}
    \hline
    Number of servers ($N$) & 10 & 100 & 1000 &$+\infty$\\\hline
    Average queue length ($m^N$) &2.81&2.39&2.36& 2.35\\\hline
    Error ($m^N-m^\infty$) & 0.45 & 0.039 & 0.004 & 0 \\\hline
  \end{tabular}
  \caption{Average queue length for the two-choice model. %  The value for $N=+\infty$ is the
    % mean-field approximation.
  }
  \label{tab:power2}
\end{table}


\bibliographystyle{ACM-Reference-Format}
\bibliography{biblio}

\end{document}